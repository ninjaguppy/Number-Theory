\chapter{Congruence}\label{ch:Three}

\section{Elementary Observations}%
\label{sec:elob}

History: Diophantus asked when equations in \(x\) with integer coefficients have
integer solutions. For example, does \(x^2-117x+31=0\)? You solves my looking at
the sign of the polynomial for even and odd values of \(x\). We will show in
this chapter that we can exploint properties of  \(\ZZ _m\) to answer these
questions.

\section{Congruence in \(ZZ\)}%
\label{sec:cong}

\begin{definition}[Congruence]
  \label{def:cong}
  If \(a,b,m \in \ZZ\), and \(m\neq 0\), then we say that \(a\) is
  \emph{congruent to b mod m} of \(m|(b-a)\) and we write \(a \equiv b (m)\)
\end{definition}

\(\equiv\) is an equivalence relation. We will use the notation \(\overline{a}\)
for the class of integers equivalent to \(a\). As long as we fix an \(m\), we
get the following characterization:

\begin{prop}
\begin{enumerate}[(a)]
  \item \(\overline{a} = \overline{b} \) iff \(a \equiv b (m)\)
  \item \(\overline{a}\neq \overline{b}\) iff
        \(\overline{a} \cap \overline{b} \) is empty
  \item There are exactly \(m\) distance congruence classes mod \(m\).
\end{enumerate}
\end{prop}

Why do we care? Well it turns out that the results allows us to turn \(\ZZ_m\)
into a ring! Not only that, the standard projection \(\ZZ \to \ZZ _m\) is a
homomorphism so it doesn't matter ``when'' we do the modulo! Even
better---things cross over into the polynomial ring. If \(p \in \ZZ [x]\) and
\(p(0)\) and \(p(1)\) are both odd, then \(p\) can't have any integer roots!

\section{On \(ax \equiv b (m)\)}%
\label{sec:axeqb}
How do we know when \(ax \equiv b (m)\) is solvable? How many solutions should
we expect it to have? The idea of a ``number'' of solutions may seem strange,
but if we have \(f(x_1, \dots , x_n)\equiv 0 (m)\), then we consider
\((a_1, \dots,a_n) \sim (b_1, \dots,b_n)\) if they are the same class in
\({\ZZ_m}^n\). For example, \(3,8,13,18\) all solve \(6x\equiv 3 (15)\), but
\(3\equiv 18 (15)\), so we say there are only 3 solutions.

Let \((a,m) = d>0\)  and \(a'=\frac{a}{d}\), \(m'=\frac{m}{d}\)---note that
this means that \(a'\) and \(m'\) are coprime

\begin{prop}
  The congruence \(ax \equiv b (m)\) has solutions if and only if \(d \mid b\).
  If \(d\mid b\), then there are exactly \(d\) solutions. If \(x_0\) is a
  solution, then the other solutions are given by \(x_0+m', x_0+2m', \dots x_0+(d-1)m'\).
\end{prop}

\begin{proof}
  If \(x_0\) is a solutins, then \(a x_0-b =m y_0\) for some \(y_0 \in \ZZ \).
  Thus, \(a x_0 + m y_0=b\). \(d\) divides the LHS, so it must divide the RHS.

  Conversely, if \(d \mid b\), then there must be integers \(x_0', y_0'\) \st
  \(a x_0'-m y_0' = d\). Let \(c=\frac{b}{d}\) and we have
  \(a(x_0'c)-m(y_0'c)=b\)---but if we let \(x_0=x_0'c\), we are done!

  Now suppose that \(x_0,x_1\) are solutions.
  \(a x_0\equiv b, a x_1\equiv b \implies a(x_1-x_0)\equiv 0\), and so
  \(m\mid a(x_1-x_0)\) \emph{and} \(m'\mid a'(x_1-x_0)\). But we have already
  stated \(m'\) and \(a'\) are coprime, so \(m'|x_1-x_0\) and thus
  \(x_1=x_0+km'\). The bounds on \(k\) are immediate.
\end{proof}

Now this proposition has two immediate (and important) consequences when it comes
to equivalences with unique solutions

\begin{corollary}
  If \(a\) and \(m\) are coprime, then \(ax \equiv b (m)\) has exactly one solution
\end{corollary}

\begin{corollary}
If \(p\) is prime and \(a \not\equiv 0(p)\), then \(ax\equiv b (p)\) has exactly
one solution.
\end{corollary}

We can interpret these props and corrs in as ring theoretic statements. Recall
that the units in \(\ZZ_m\) are the invertable elements and for a multiplicative
group. In particular, \(a \in \ZZ _m\) is a unit iff \(ax=1\) is
solvable---which happens precisely when \(a\) and \(m\) are coprime. It follows
that there are \(\phi(m)\) units in \(\ZZ_m\).

In the special case that \(p\) is
prime, all nonzero elements of \(\ZZ_p\) are coprime to \(p\) and hence units.
Thus, \(\ZZ _p\) is field! If \(m\) is not prime, then \(m=m_1m_2\). But then in
\(\ZZ _p\), \(\overline{m_1} \overline{m_2}=\overline{0}\)---so we are not even
a domain, let alone a field. To summarize,

\begin{prop}
  An element \(\overline{a} \in \ZZ _m\) is a unit iff \((a,m)=1\). There are
  exactly \(\phi(m)\) units in \(\ZZ _m\). Further, \(\ZZ _m\) is a field iff
  \(m\) is prime.
\end{prop}

\begin{corollary} [Euler's Theorem]
If \((a,m)=1\), then \(a^{\phi(m)}\equiv 1(m)\)
\end{corollary}

\begin{proof}
  We know that the units of \(\ZZ _m\) form a group of order \(\phi(m)\). Since
  \(a\) is a unit, Lagagrange's theorem tells us that
  \(\overline{a}^{\phi(m)}=\overline{1}\) \(\implies a^{\phi(m)}\equiv (m)\).
\end{proof}

\begin{corollary} [Fermat's Little Theorem]
  If \(p\) is prime and \(p \nmid a\), then \(a^{p-1}\equiv 1 (p)\).
\end{corollary}

\begin{proof}
  The above corr and the fact that for \(p\) prime we have that \(\phi(p)=p-1\).
\end{proof}


\section{The Chinese Remainder Theorem}%

When our \(m\) is composite, we can sometimes simplify our correspondence so a
system of congruences. Indeed, such a simplification is possible in a much
larger class of rings (the PID case is in the selected problems for this
chapter).

\begin{lemma}
If \(a_1, \dots , a_n\) are all relatively prime to \(m\), then so is their product.
\end{lemma}

\begin{proof}
  \(U(\ZZ_m)\) is a group, and hence closed under multiplication.
\end{proof}

\begin{lemma}
Suppose \(a_1, \dots a_t\) all divide \(n\) and that the \(a_i\) are pairwise
coprime. Then \(a_1\cdots a_t\mid n\).
\end{lemma}
\begin{proof}
  We proceed by induction on \(t\). \(t=1\) is immediate. Now assume that our
  lemma is true for \(t-1\), so \(a_1\cdots a_{t-1} \mid n\). By the previous
  lemma, \(a_t\) is prime to \(a_1\cdots a_{t-1}\). It follows that there are
  \(r,s \in \ZZ \) \st \(ra_t + sa_1\cdots a_{t-1} =1\). If we multiply both
  sides by \(n\), see that the LHS is divisible by \(a_1\cdots a_t\)---so \(n\)
  must be as well!
\end{proof}

Now we state the big boy theorem feared by many an undergraduate:

\begin{theorem} [Chinese Remainder Theorem]
Suppose \(m=m_1\cdots m_t\) where the \(m_i\) are pairwise coprime. Let
\(b_1, \dots , b_t \in \ZZ \) and consider the system
\[
  x \equiv b_1(m_1), x \equiv b_2 (m_2), \dots ,x \equiv b_t (m_t)
\]
This system always has solutions and any two solutions differ by a multiple of \(m\).
\end{theorem}

\begin{proof}
  Let \(n_i= \frac{m}{m_i}\). From two lemmas ago (using \verb|\ref| is too much
  work) we have that \((m_i,n_i)=1\). Thus, we have \(r_i,s_i \in \ZZ \) \st
  \(r_i m_i+s_i n_i=1\). Let \(e_i=s_i n_i\)---and so \(e_i\equiv 1 (m_i)\) but
  \(e_i \equiv 0 (m_j)\) for \(i\neq j\). Now define \(x_0= \sum_{i=0}^t b_i e_i\). Notice that this means that
  \(x_0 \equiv b_i e_i (m_i)\implies x_0\equiv b_i(m_i)\) and so \(x_0\) is a
  solution!\footnote{I'm not sure what do with this, but it sure smells like
    some sort of orthogonal decomp. What are the vectors? What is the inner
    product? Honestly, it might be a bit more clear in the context of the more
    general statement of the theorem---linear algebra over a PID has a ton of structure.}

  Now suppose that \(x_1\) is another solution. Then \(x_1-x_0\equiv 0 (m_i)\)
  for all \(i\). In other words, each \(m_i\mid x_1-x_0\), and so by the other
  previous lemma \(m = m_1\cdots m_t | x_1-x_0\).
\end{proof}

This has a nice ring theoretic interpretation. Recall that if we have a handful
of rings, \(R_1, \dots , R_t\), we can form a new ring
\(S = R_1 \oplus \cdots \oplus R_t\) with operations done component-wise. In
this case \(U(S) = U(R_1) \times \cdots \times U(R_t)\).

Now we turn our attention back to the ring \(\ZZ_m\). Recalling some category
theory, recall that both we have canonical maps \(\psi_i: Z\to \ZZ_{m_i}\) and
\(\alpha_i: \ZZ _{m_i} \hookrightarrow \oplus \ZZ_{m_j}\). Since \(\alpha_i\)
exists for each \(m_i\), we have an induced map:
% https://q.uiver.app/?q=WzAsMyxbMCwwLCJcXFpaIl0sWzAsMiwiXFxaWl97bV9pfSJdLFsyLDIsIlxcYmlnb3BsdXNfe21fan0gXFxaWl97bV9qfSJdLFswLDEsIlxccGhpX0kiXSxbMSwyLCJcXGFscGhhX0kiXSxbMCwyLCJcXHBoaSA9IFxccHJvZCBcXGFscGhhX0kiXV0=
\[\begin{tikzcd}
	\ZZ \\
	\\
	{\ZZ_{m_i}} && {\bigoplus_{m_j} \ZZ_{m_j}}
	\arrow["{\psi_i}", from=1-1, to=3-1]
	\arrow["{\alpha_i}", from=3-1, to=3-3]
	\arrow["{\psi = \prod \alpha_i}", from=1-1, to=3-3]
\end{tikzcd}\]

Now we consider the image and kernel of \(\psi\). By the CRT above, we have that
\(\psi\) is onto. Further, the kernel is precisely the ideal \(m\ZZ \). Thus, we
have an isomorphism between
\[
  \ZZ _m \simeq \ZZ_{m_1} \oplus \cdots \oplus \ZZ_{m_t}
\]
whenever \(m=m_1\cdots m_t\) and the \(m_i\) are pairwise coprime. This
immedately tells us about the units of \(\ZZ _m\):
\(\phi(m)=\phi(m_1)\cdots\phi(m_t)\). If we consider the prime decompositon of
\(m\), \(m= p_1^{a_1}\cdots p_t^{a_t}\), then
\(\phi(m)= \phi(p_1^{a_1})\cdots \phi(p_t^{a_t})\). Now we know that
\(\phi(p^a)=p^a-p^{a-1}=p^a(1-1 / p)\) and it follows that
\begin{align*}
  \phi(m) &= \prod p_i^{a_i}\left(1-\frac{1}{p_i}\right) \\
          &= p_1^{a_1}\cdots p_t^{a_t} \prod\left(1-\frac{1}{p_i}\right) \\
          &= m \prod\left(1-\frac{1}{p_i}\right)
\end{align*}
