\chapter{Unique Factorization}\label{ch:One}

\section{Unique Factorization in \(\ZZ \)}\label{sec:introduction}

We say that \(a\) divides \(b\) if there is some \(c\) \st \(b=ac\)---we write
\(a\mid b\). A \textbf{prime} number is whose only divisors are \(1 \text{ and
} p\). Of great
interest is the function \(\pi(x)\)---which outputs the number of primes between
1 and \(x\).

\begin{theorem}
  True facts about dividing
  \begin{enumerate}
    \item \(a \mid  a\) for \(a\neq 0\).
    \item If \(a \mid b\) and \(b\mid a\), then \(a=\pm b\)
    \item If \(a\mid b, b\mid c\), then \(a\mid c\)
    \item If \(a \mid b, a\mid c\), then \(a \mid \alpha b+ \beta c\)
  \end{enumerate}
\end{theorem}

\begin{definition}[ord\(_p\)]
  \label{def:ord}
  Let \(n \in \ZZ \), \(p\) a prime. Then \(\ord(n)=a\) where \(a\) is the
  largest integer such that \(p^a\mid n\) but \(p^{a+1}\nmid n\). Define \(\ord (0)=\infty\).
\end{definition}
%
\(\ord\) gives the number of times that \(p\) divides \(n\)---the power of \(p\)
in \(n\)'s prime decomposition. On that note, every \(n \in \ZZ \) has a unique
prime decomposition:
\begin{theorem}
  For every nonzero \(n \in \ZZ \), there is a prime decomposition
  \[
    n = (-1)^{\varepsilon(n)} \prod_p p^{a(p)}
  \]
  with the exponents uniquely determined by \(n\). In fact, we have that
  \(a(p)=\ord n\).
\end{theorem}

See that it must be the case that all but finitely many of \(a(p)\) must be
0---or else we have some problems with \(n\). The fact that \(\ZZ \) is a PID
plays nicely with the following definition

\begin{definition}[GCD]
  \label{def:gcd}
  Let \(a,b \in \ZZ \). An integer \(d\) is called the gcd of \(a\) and \(b\) if
  every other common divisor of \(a\) and \(b\) divides \(d\)
\end{definition}

You can use theorem 1 to show that the gcd is determined up to sign---so we can
speak of \emph{the} gcd. Even better (if you know some rings) is the following

\begin{lemma}
  Let \(a,b \in \ZZ \). If \((a,b) = (d)\), then \(d\) is the gcd of \(a\) and \(b\).
\end{lemma}

We say that \(a,b\) are \textbf{relatively prime} is their only common divisors
are \(\pm 1\). It is worth noting that if \(a | bc\) and \((a,b) = 1\) (where
\((a,b)\) denotes the gcd), then it must be the case that \(a|c\). This has two
important consequences

\begin{corollary}
  If \(p\) is prime and \(p | bc\), then either \(p| b\) or \(p | c\). (notice
  the hook into prime ideals)
\end{corollary}

\begin{corollary}
  For \(p\) prime and \(a,b \in \ZZ \), \(\ord ab = \ord a + \ord b\).
\end{corollary}

\section{Unique Factorization in \(k[x]\)}%
\label{sec:1.2}

This section retells the story of section 1 but in the case of a polynomial ring
over a field --- nothing to note if I'm being honest.

\section{Unique Factorization in a PID}%
\label{sec:1.2}

\begin{definition}[Euclidean Domain]
  \label{def:Edom}
  If \(R\) is a commutative domain, it is a Euclidean domain if there is a
  function \(\lambda: R^+ \to \NN \cup \{0\} \) such that if
  \(a,b \in R, b \neq 0\), then there exists \(c,d \in R\) with the property
  that \(a=cb+d\) and either \(d=0\) or \(\lambda(d) < \lambda(b)\).
\end{definition}

Both \(\ZZ \) and \(k[x]\) (for \(k\) a field) are EDs. In \(\ZZ \) we can use
the standard absolute value function. For \(p \in k[x]\), use \(\lambda(p)=\textrm{deg}(p)\).

In particular, it can be proven that every ED is a PID---the converse is not
true but there aren't many readily available examples. Because of this nice
structure, we inherit the same notion of divisibility. We say an element
\(u \in R\) is a \textbf{unit} if \(u | 1\)---\ie units have a mult.\ inverse. Two
elements \(a,b \in R\) are \textbf{associate} if \(a=bu\) for some unit \(u\).
An element \(p \in R\) is \textbf{irreducible} if \(a|p\) means that \(a\) is a
unit or an associate of \(p\). A nonunit \(p \in R\) is prime if \(p \neq 0\)
and \(p|ab \implies p|a\; \textrm{ or } p|b\). Whiles these seem different, in a
general PID we have that prime \(\Leftrightarrow\) irreducible.

As seen in the previous two sections, our theory becomes richer when we utilize
the language of ideals. Indeed, \(a\mid b \Leftrightarrow (b) \subset (a)\),
\(u \in R\) is a unit \(\Leftrightarrow (u) = R\) (since \(u\) has a
multiplicative inverse), \(a,b\) are associate \(\Leftrightarrow (a)=(b)\)
(simply multiply by the correct unit), \(p\) is prime
\(\Leftrightarrow ab \in (p) \implies a \in (p) \textrm{ or } b \in p )\).
Even better, while the concept of a gcd does not exist in a general ring, since
we are in a PID, we get the follwing resul:

\begin{prop}
  Let \(R\) be a PID and \(a,b \in R\). Then \(a\) and \(b\) have a greatest
  common divisor \(d\), and \((a,b) = (d)\).
\end{prop}

\begin{proof}
  Consider \((a,b)\)---since \(R\) is a PID, there is an element \(d\) s.t.\
  \((a,b)=(d)\). Further, \((a) \subset(d)\) and \((b)\subset(d)\) so
  \(d \mid a, d \mid b\). Now suppose \(d'\mid a, d'\mid b\). Then
  \((a) \subset(d'), (a) \subset(d')\). And so it must be the case that
  \((d) = (a,b) \subset(d')\)---and hence \(d \mid d'\), so \(d\) is our gcd.
\end{proof}

\begin{remark}
  Note that in the above case we can have \((d)=(d')\), but then \(d\) and
  \(d'\) are associates---and we only care about gcd up to associates (in the
  same way that in \(\ZZ \) we have a \(\pm\) problem if we look for a ``totally
  unique'' gcd.)
\end{remark}

From this proposition, we get two immediate corollaries---if \(a,b\) are
relatively prime, then \((a,b) = R\) (relatively prime means you can make a
linear combination to get 1) and that irreducible \(\implies \) prime.

We now endevor to prove that every nonzero \(a \in R\) is a product of
irreducible elements---we use the ascending chain condition!

\begin{lemma}
  A PID obeys the ASC.
\end{lemma}

\begin{proof}
  Let \((a_1) \subset(a_2) \subset \dots\) be an ascending chain of ideals in \(R\), and
  let \(I\) be the union along the chain. Since \(R\) is a PID, \(I = (a)\)
  for some \(a \in R\). But then \(a\) is in the union along a chain, so it must
  be in one of the \((a_k)\)---so \(I \subset(a_k)\) and we have that chain must
  be eventually constant!
\end{proof}

\begin{prop}
  Every nonzero nonunit of a PID is a product of irreducibles.
\end{prop}

\begin{proof}
  First we show that \(a\) is divisible by an irreducible. For there to be any
  meat, suppose that \(a\) is \emph{not} irreducible. But then \(a=a_1b_1\) for
  \(a_1,b_1\) nonunits. If \(a_1\) is irrecducible, we are done! If it is not,
  then we can factor \(a_1=a_2b_2\). Continuing, we get a chain
  \((a) \subset(a_1) \subset(a_2) \subset \cdots\)---which must be eventually
  constant, giving us the required \(a_k\), an irreducible which divides \(a\).

  Now we show that we can break down \(a\) entirely into irreducibles. For a
  nonirreducible \(a\), we have some irred \(p_1\) s.t. \(p_1\mid a\), so we can
  write \(a = p_1c_1\). If \(c_1\) is a unit, we are done. If it is not we can
  factor further as \(c_1=p_2c_2\). Continuing, we get a chain
  \((c) \subset(c_1) \subset(c_2) \subset \cdots\)---which must be eventually
  constant, giving us the required \(c_k\) which is a unit. Thus, we can writ
  \(a=p_1p_2\cdots p_kc_k\), a product of irreducibles.
\end{proof}

Now this ASC is incredibly powerful. In fact, we use it to prove the following
Lemma that will help us reproduce an analogue of the ord function in a general
PID.

\begin{lemma}
  Let \(p\) be a prime and \(a\neq 0\). Then there is an integer \(n\) such that
  \(p ^{n} \mid a\) but \(p ^{n+1} \nmid a\).
\end{lemma}

Since this \(n\) is uniquely determined, we set \(n = \ord_{p} a\). It is not
difficult to show that \(\ord\) obeys the same multiplicative \(\to \) additive
identity that it does in the \(\ZZ \) and \(k[x]\). Now we seek to take the
previous proposition a bit further and show uniqueness of the decompostion. To
do so, we need to deal with the case of associate primes. Let \(S\) be a set of
primes in \(R\) such that
\begin{enumerate}
  \item Every prime in \(R\) is associate to a prime in \(S\)
  \item No two primes in \(R\) are associate
\end{enumerate}

\begin{theorem}
  Let \(R\) be a PID and \(S\) a set of primes as discussed above. Then if
  \(a \in R, a \neq 0\), then we can write
  \[
    a = u \prod_{p \in S} p^{e(p)}
  \]
  where \(u\) is a unit. The units and exponents are uniquely determined and
  \(e(p)=\ord_p a\).
\end{theorem}

\begin{proof}
  Existence is already proven. Uniqueness follows directly from the properties
  that we have already stated about \(\ord\).
\end{proof}

\begin{remark}
  Section 1.4 is uninteresting. Let \(\omega\) be the primitive third root of
  unity. Then \(Z[i]\) and \(\ZZ [\omega]\) are Euclidean Domains. For
  \(\ZZ [i]\), \(\lambda(a+bi)=a^2+b^2\). For
  \(\ZZ [\omega], \lambda(\alpha)=\lambda(a+b\omega)=\alpha \overline{\alpha}\).
\end{remark}
