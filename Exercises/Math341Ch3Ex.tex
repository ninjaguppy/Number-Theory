\chapter*{Chapter 3 Selected Problems}

\paragraph{Problem 1:} Show that there are infinitely many primes congruent to
\(-1\) modulo \(6\).

\begin{proof}
  Before we show that there are inifinetly many primes of this form, we need to
  show that there are \emph{any} primes of the for \(6k-1\).
  Division by 6 gives remainder of \(-1,0,1,2,3,4\)---so any odd number is
  necessarily of the form \(6k+1, 6k+3, 6k-1\). Some algebra gives that
  \begin{align*}
    (6k+1)(\overline{6k}+1) &= 6(6k \overline{k}-k-\overline{k})+1 \\
    (6k+3)(\overline{6k}+3) &= 6(6k \overline{k}-3k-3\overline{k}+1)+3 \\
  \end{align*}
  and so any odd number of the form \(6k-1\) must be divisible by a prime of the
  form \(6k-1\)---otherwise how did we leave the closed systems of \(6k+1\) and
  \(6k+3\).

  Not suppose that there were a finite list of positive primes, \(p_1, \dots ,p_n\) of
  the form \(p_i=6k_i-1\) and set \(N= 6 \prod_i p_i -1\). Now \(N\) can be
  divisible by any of the \(p_i\) but is of the form discussed above---so our list
  must be incomplete! Therefore there must be infinite primes of the for
  \(6k-1\) and we are done!
\end{proof}

\paragraph{Problem 4:} Show that the equation \(3x^2+2 = y^2\) has no solutions
in the integers.

\begin{proof}
  Seeking to use the result from the text, we set \(f(x,y) = 3x^2-y+2\) and
  see what happens when we plug in odd or even numbers:
  \begin{align*}
    f(0,0) &= 2 \\
    f(1,0) &= 3+2=5 \\
    f(0,1) &= -1+2=1 \\
    f(1,1) &= 3-1+2=4
  \end{align*}
  This rules out solutions of the form \((e,o)\) and \((o,e)\). But tells us
  nothing about the cases of \((e,e)\) or \((o,o)\). The natural next step would
  be to check solutions mod 4. Notice that this only helps us in the case that
  we are \((e,e)\)---as the \(o,o\) case is equal to 4. Before we deal with
  that, we show there can't be any solutions of the for \((e,e)\) by show that
  \(f(e,e)\) is never 0 mod 4
  \begin{align*}
    f(0,0) &= 2\\
    f(2,0) &= 0-2^2+2 \equiv 2 \\
    f(0,2) &= 3\cdot 2^2-0+2 \equiv 2 \\
    f(2,2) &= 3\cdot 2^2-2^2+2 \equiv 2 \\
  \end{align*}
  Since the \((e,e)\) case always gives us 2 (mod 4), we can't have a solution
  of the form \((e,e)\).

  Now we turn our attention back to the case of \((o,o)\). We will check these
  solutions mod 3. There are 4 cases to check, but since we consider \(3\) to be
  0, we can use our previous computation:
  \begin{align*}
    f(0,0) &= 2 \equiv 2 (3) \\
    f(1,0) &= 3+2=5 \equiv 1 (3)\\
    f(0,1) &= -1+2=1 \equiv 1 (3)\\
    f(1,1) &= 3-1+2=4\equiv 1 (3)
  \end{align*}
  And so any  solution of the form \((o,o)\) fails mod 3 and we have
  exhausted all the possible cases!
\end{proof}

\paragraph{Problem 12:} Show that if \(p\) is prime and \(1\leq k\leq p-1\),
then \(p \mid  \binom{p}{k}\). Deduce that \((a+1)^p \equiv a^p+1 (p)\).

\begin{proof}
  See that if \(1\leq k\leq p-1\), then
  \[
    \binom{p}{k} = \frac{1 \cdot 2 \cdots p}{(1\cdot 2 \cdots k)(1 \cdot 2 \cdots (p-k))}
    = \frac{(p-k) \cdot  (p-k+1) \cdots p}{1 \cdot 2 \cdots (p-k)}
  \]
  but since \(p\) is prime and \(p\) does not appear in the denominator, \(p\mid \binom{p}{k}\)

  For the second part, consider the fact that
  \begin{align*}
    (a+1)^p &= \sum_{k=0}^p \binom{p}{k} a^{b-k}.
  \end{align*}
  By the above work, everything except for the \(k=0,p\) drops to 0 mod \(p\).
  The \(k=0\) gives \(a^p\) and the \(k=p\) case gives \(1\). Thus, we have
  recovered the freshmans dream:
  \[
    (a+1)^p \equiv a^p+1 (p)
  \]
\end{proof}

\paragraph{Problem 17 :} Let \(f \in\ZZ[x]\) and
\(n=p_1^{a_1}\cdots p_t^{a_t}\). Show that \(f(x) \equiv 0 (n)\) has a solution
iff \(f(x) \equiv 0 (p_i^{a_i})\) has a solution for each \(i\).

\begin{proof}
  (\(\implies \)) This direction is obvious. Since \(p_i^{a_i}\mid n\) for all
  \(i\), if \(n\mid f(x)\), then \(p_i^{a_i}\mid f(x)\).

  (\(\impliedby \)) We know that \(p_1^{a_1} \mid  f(x)\), so
  \(f(x) = p_1^{a_1}k_1\). But since \(p_2^{a_2} \mid f(x)\) and
  \(p_2^{a_2} \nmid p_1^{a_1}\), we know that \(f(x)=p_1^{a_1}p_2^{a_2}k_2\).
  Continuing for all the \(p_i\), we have that
  \(f(x)=p_1^{a_1}\cdots p_t^{a_t}k_t\). It follows immediately that \(n|f(x)\).
\end{proof}

\paragraph{Problem 18:} Let \(N\) be the number of solutions to
\(f(x) \equiv 0 (m)\) and \(N_i\) the number of solutions to
\(f(x) \equiv 0 (p_i^{a_i})\). Show that \(N = \prod_{i}  N_i\).

\begin{proof}
  As seen above, a solution to \(f(x) \equiv 0 (n)\) can be constructed from a
  product of solutions to solutions to \(f(x) \equiv 0 (p_i^{a_i})\). If we view
  this product as a \(t\)-tuple of integers \(x_i\), where each \(x_i\) solve
  the related equivalence (to \(p_i^{a_i}\)), then our solution is given by the
  map \((x_1, \dots , x_t) \mapsto \prod x_i\). There are \(N_i\) solutions for
  each entry, so some basic combinatorics tells us that there are
  \(\prod_i N_i\) possible combinations, and hence \(\prod_i N_i\) possible solutions!
\end{proof}

\paragraph{Problem 22:}  Formulate and prove the Chinese Remainder Theorem for a
PID. I was actually able to prove it for any commutative ring with unity!

\begin{proof}[Solution:]
  First we need a concept of coprime ideals. Given two ideals \(I,J \subset R\),
  we say that they are \textbf{coprime} if these are elements
  \(r \in I, s \in J\) such that \(r+s = 1\)---Notice that this says that
  \(I+J = R\). It is will known that \(IJ \subset I \cap J\), but the opposite
  containment can occasionally hold:

  \begin{nlemma}
    If \(R\) is any commutative ring with \(1\)  and \(I,J\) are coprime ideals,
    then \(IJ = I \cap J\).
  \end{nlemma}

  \begin{proof}
    As mentioned above, \(\subset\) is well known, so let \(a \in I \cap J\).
    Since \(I,J\) are coprime, we have that there are \(r\in I, s \in J\) such
    that \(r+s=1\). We can write \(a=a(r+s)=ar+as\)---this a linear combination
    of \(r\) and \(s\), so it lives in \(IJ\), hence \(a \in IJ\).
  \end{proof}

  With our lemma proven, we can proceed. For those fearful of category theory,
  we recommend that you look away because we invoke the universal property of
  products: given a product \(\prod X_i\) and any set of maps
  \(f_i: Y \to X_i\), there is a unique map \(f\) such that the following
  diagram commutes for every \(i\):
  % https://q.uiver.app/?q=WzAsMyxbMCwwLCJZIl0sWzAsMiwiWF9JIl0sWzIsMiwiXFxwcm9kX0kgWF9JIl0sWzAsMSwiZl9JIl0sWzIsMSwiXFxwaV9JIiwyXSxbMCwyLCJmIiwwLHsic3R5bGUiOnsiYm9keSI6eyJuYW1lIjoiZGFzaGVkIn19fV1d
\[\begin{tikzcd}
	Y \\
	\\
	{X_i} && {\prod_i X_i}
	\arrow["{f_i}", from=1-1, to=3-1]
	\arrow["{\pi_i}"', from=3-3, to=3-1]
	\arrow["f", dashed, from=1-1, to=3-3]
  \end{tikzcd}\]
  With this background, we can finally state and prove our theorem.
  \begin{ntheorem}
   Let \(I_1, \dots , I_n\) be pairwise coprime ideals in \(R\), a commutative
   ring with unity and \(I = \prod_i I_i\). Then the natural map
   \[
     \varphi: \faktor{R}{I} \to \prod_i \faktor{R}{I_i}
   \]
   is an isomorphism.
  \end{ntheorem}
  \begin{proof}
    Let \(\rho_i: R\to \faktor{R}{I_i} \) and
    \(\pi: \prod_i \faktor{R}{I_i} \to \faktor{R}{I_i} \) be the standard
    quotient and projection maps. Invoking the universal property of products,
    we have a map \(\varphi\) such that the following diagram commutes for all \(i\):
    % https://q.uiver.app/?q=WzAsMyxbMCwwLCJSIl0sWzAsMiwiXFxmYWt0b3J7Un17SV9JfSJdLFsyLDIsIlxccHJvZF9JIFxcZmFrdG9ye1J9e0lfSX0iXSxbMCwxLCJcXHJob19JIl0sWzIsMSwiXFxwaV9JIiwyXSxbMCwyLCJcXHZhcnBoaSJdXQ==
\[\begin{tikzcd}
	R \\
	\\
	{\faktor{R}{I_i}} && {\prod_i \faktor{R}{I_i}}
	\arrow["{\rho_i}", from=1-1, to=3-1]
	\arrow["{\pi_i}"', from=3-3, to=3-1]
	\arrow["\psi", from=1-1, to=3-3]
\end{tikzcd}\]
    Category also guarantees that \(\psi\) is a ring homomorphism, so there
    is no need to check that! Further, it must be the case that \(\psi\) is
    onto. See that both \(\pi_i\) and \(\rho_i\) are onto. This, combined with
    the fact that \(\pi_i\) is a projection onto \(\faktor{R}{I_i} \) and the
    commutativity of the diagram gives us that  \(\psi\) must be onto as well.

    What is the kernel of \(\psi\)? It is precisely the elements which are in
    each of the \(I_i\) (as the must be zero in each component)---meanning the
    kernel is \(\bigcap_i I_i= \prod_i I_i\) (by the lemma). Thus by the first
    isomorphism theorem, \(\psi\) descends to an isomorphism.
   \[
     \varphi: \faktor{R}{I} \to \prod_i \faktor{R}{I_i}
   \]
  \end{proof}
\end{proof}
